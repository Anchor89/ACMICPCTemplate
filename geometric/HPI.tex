\subsection{半平面交}
    直线左边代表有效区域。
    \begin{lstlisting}[language=c++]
bool HPIcmp(Line a, Line b)
{
    if (fabs(a.k - b.k) > eps)    return a.k < b.k;
    return ((a.s - b.s) * (b.e-b.s)) < 0;
}

Line Q[100];
void HPI(Line line[], int n, Point res[], int &resn)
{
    int tot = n;
    sort(line, line + n, HPIcmp);
    tot = 1;
    for (int i = 1; i < n; i++)
        if (fabs(line[i].k - line[i - 1].k) > eps)
            line[tot++] = line[i];
    int head = 0, tail = 1;
    Q[0] = line[0];
    Q[1] = line[1];
    resn = 0;
    for (int i = 2; i < tot; i++)
    {
        if (fabs((Q[tail].e-Q[tail].s) * (Q[tail - 1].e-Q[tail - 1].s)) < eps ||
                fabs((Q[head].e-Q[head].s) * (Q[head + 1].e-Q[head + 1].s)) < eps)
            return;
        while (head < tail && (((Q[tail]&Q[tail - 1]) - line[i].s) * (line[i].e-line[i].s)) > eps)
            tail--;
        while (head < tail && (((Q[head]&Q[head + 1]) - line[i].s) * (line[i].e-line[i].s)) > eps)
            head++;
        Q[++tail] = line[i];
    }
    while (head < tail && (((Q[tail]&Q[tail - 1]) - Q[head].s) * (Q[head].e-Q[head].s)) > eps)
        tail--;
    while (head < tail && (((Q[head]&Q[head + 1]) - Q[tail].s) * (Q[tail].e-Q[tail].s)) > eps)
        head++;
    if (tail <= head + 1) return;
    for (int i = head; i < tail; i++)
        res[resn++] = Q[i] & Q[i + 1];
    if (head < tail + 1)
        res[resn++] = Q[head] & Q[tail];
}
    \end{lstlisting}
